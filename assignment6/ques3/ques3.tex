\documentclass{article}
\usepackage{amsthm}
\usepackage{amsmath}
\usepackage{tikz}
\usetikzlibrary{positioning, arrows,automata}
\usetikzlibrary{shapes}
\usepackage{dot2texi}
\begin{document}
\section{ Kuratowski graphs K(5) }

\begin{tikzpicture}[thick,every state/.style={draw=black,very thick,fill=blue!30}]
\begin{dot2tex}[neato,mathmode]
  graph G {
    node[style="state", label="" ]
    1 -- 2;
    1 -- 3;
    1 -- 4;
    1 -- 5;
    2 -- 3;
    2 -- 4;
    2 -- 5;
    3 -- 4;
    3 -- 5;
    4 -- 5;
  }
\end{dot2tex}
\end{tikzpicture}
\section{ Kuratowski graphs K(3,3) }
\begin{tikzpicture}[thick,every state/.style={draw=black,very thick,fill=blue!30}]
\begin{dot2tex}[mathmode]
  graph G {
    node[style="state", label="" ]
    1 -- 4;
    1 -- 5;
    1 -- 6;
    2 -- 4;
    2 -- 5;
    2 -- 6;
    3 -- 4;
    3 -- 5;
    3 -- 6;  
  }
\end{dot2tex}
\end{tikzpicture}

\section{Petersen's graph}
\begin{tikzpicture}[thick,every state/.style={draw=black,very thick,fill=blue!30}]
\begin{dot2tex}[neato,mathmode]
  graph G {
    node[style="state", label="" ]
      	1 -- 2 -- 3 -- 4 -- 5 -- 1 [len = 40];
	1 -- 6 ;
	2 -- 7 ;
	3 -- 8 ;
	4 -- 9 ;
	5 -- 10;
	6 -- 8 -- 10 -- 7 -- 9 -- 6;
  }
\end{dot2tex}
\end{tikzpicture}

\end{document}
